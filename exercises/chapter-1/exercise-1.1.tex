\documentclass{article}

\usepackage[margin=1in]{geometry}
\usepackage{amsmath}
\usepackage{hyperref}

\setlength{\parindent}{0pt}
\setlength{\parskip}{1em}

\begin{document}

\section*{Exercise 1.1}

For each of the mechanical systems described below, give the number of
degrees of freedom of the configuration space.

\section*{1.1 (a)}

Three juggling pins.

Rigid bodies generally have 6 degrees of freedom: three parameters to
specify their position in space and three to specify their
orientation.

Since we have three pins, the number of degrees of freedom is
$6 \times 3 = \boxed{18}$.

\section*{1.1 (b)}

A spherical pendulum, consisting of a point mass (the pendulum bob)
hanging from a rigid massless rod attached to a fixed support point.
The pendulum bob may move in any direction subject to the constraint
imposed by the rigid rod. The point mass is subject to the uniform
force of gravity.

A point mass generally has three degrees of freedom, corresponding to
its position in space. The rod acts as a constraint, forcing it to
move on a spherical surface. Therefore, the number of degrees of
freedom are $3 - 1 = \boxed{2}$.

These correspond to the two angles you need to know ($\theta$ (polar)
and $\phi$ (azimuth)) to locate the pendulum bob.
\href{https://en.wikipedia.org/wiki/Spherical_pendulum}{Wikipedia has
  a good diagram of these angles.}

\section*{1.1 (c)}

A spherical double pendulum, consisting of one point mass hanging from
a rigid massless rod attached to a second point mass hanging from a
second massless rod attached to a fixed support point. The point
masses are subject to the uniform force of gravity.

This is just the double of the previous answer, as we have two point
masses and two constraints. $2 \times (3 - 1) = \boxed{4}$.

\section*{1.1 (d)}

A point mass sliding without friction on a rigid curved wire.

Since we know about the shape of the curved wire, the point mass is
simply moving in a 1D curve in 3D space. Therefore all we we have is
\fbox{1} degree of freedom, which is how far along the wire it is.

\section*{1.1 (e)}

A top consisting of a rigid axisymmetric body with one point on the
symmetry axis of the body attached to a fixed support, subject to a
uniform gravitational force.

Since one point of the top is fixed, its translational freedom goes
away leaving us with only three degrees of freedom for its
orientation. Further, because it's axisymmetric, it loses one of these
degrees of freedom as you cannot discern rotations about its axis of
symmetry.

Therefore, we this has $3 - 1 = \boxed{2}$ degrees of freedom. These
correspond to the two angles needed to describe the orientation of its
axis of symmetry.

\section*{1.1 (f)}

The same as (e), but not axisymmetric.

Because it's not axisymmetric, we don't lose that one rotational
degree of freedom (rotation about its axis of symmetry) we subtracted
above. So this case has \fbox{3} degrees of freedom. These correspond
to two angles needed to specify the orientation of its axis of
rotation, and one to specify its rotation about that axis.

\end{document}
