\documentclass{article}

\usepackage[margin=1in]{geometry}
\usepackage{amsmath}

\setlength{\parindent}{0pt}
\setlength{\parskip}{1em}

\begin{document}

\section*{Exercise 9.1}

We're given the following functions and asked about some of their
derivatives.
\begin{align*}
  F(x, y) &= x^2 y^3 \\
  G(x, y) &= (F(x, y), y) = (x^2 y^3, y) \\
  H(x, y) &= F(F(x, y), y) = F(x^2 y^3, y) = (x^2 y^3)^2 y^3 = x^4 y^9
\end{align*}
In the calculus that follows, you'll notice that I map back to
Leibniz's notation. While this feels like it goes against the spirit
of the book, it's helpful until you're comfortable with functional
notation.

\section*{9.1 (a)}

\begin{align*}
  \partial_o F(x, y) &= \frac{\partial}{\partial x} x^2 y^3  = 2 x y^3 \\
  \partial_1 F(x, y) &= \frac{\partial}{\partial y} x^2 y^3  = 3 x^2 y^2 \\
\end{align*}

\section*{9.1 (b)}

\begin{align*}
  \partial_o F(F(x, y), y) &= \partial_o H(x, y) = \frac{\partial}{\partial x} x^4 y^9 = 4 x^3 y^9 \\
  \partial_1 F(F(x, y), y) &= \partial_1 H(x, y) = \frac{\partial}{\partial y} x^4 y^9 = 9 x^4 y^8
\end{align*}

\section*{9.1 (c)}

\begin{align*}
  \partial_o G(x, y) &= \frac{\partial}{\partial x} (x^2 y^3, y)  = (2 x y^3, 0) \\
  \partial_1 G(x, y) &= \frac{\partial}{\partial y} (x^2 y^3, y)  = (3 x^2 y^2, 1) \\
\end{align*}
Notice that unlike $F$ and $H$ which return numbers, $G$ returns a
tuple containing $F$. So its partial derivatives are also tuples, with
one set of derivatives matching those computed for $F$ in 9.1 (a).

\section*{9.1 (d)}

To arrive at the following, we group the partial derivatives
determined previously into tuples, and then substitute the given
inputs.
\begin{align*}
  DF(a, b) &= [2 a b^3, 3 a^2 b^2] \\
  DG(3, 5) &= [(2 \cdot 3 \cdot 5^3, 0), (3 \cdot 3^2 \cdot 5, 1)] \\
           &= [(750, 0), (675, 1)] \\
  DH(3 a^2, 5 b^3) &= [4 (3 a^2)^3 (5 b^3)^9, 9 (3 a^2)^4 (5 b^3)^8] \\
                   &= [210937500 a^6 b^{27}, 284765625 a^8 b^{24}]
\end{align*}

\end{document}
